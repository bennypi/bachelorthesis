\chapter{Einleitung}

In der Robotik werden bildgebende Sensoren benutzt, um Informationen über das Umfeld des Roboters zu erhalten. Diese Informationen können genutzt werden, um Hindernisse zu erkennen, die der Roboter meiden soll oder es werden Gegenstände gesucht, die der Roboter greifen und mit ihnen interagieren soll. Es gibt verschiedene Arten von bildgebenden Sensoren, wie z.B. Kameras, die ein zweidimensionales Bild erzeugen und Kameras, die zusätzlich zu dem zweidimensionalen Bild auch Tiefeninformationen erkennen. Andere Sensoren, die ein räumliches Abbild ohne Farbinformationen erstellen, sind z.B. Lidar-, Ultraschall- und Radarsensoren.

Bei all diesen Sensoren ist es wichtig, dass diese korrekt kalibriert sind, damit die gemessenen Daten der Realität möglichst nahe kommen. Stimmen die Messdaten nicht mit der Realität überein, stößt der Greifarm ein Objekt möglicherweise um, anstatt es zu greifen.

Kamerasensoren, die nach dem Lochkameraprinzip arbeiten, werden durch die Kameramatrix definiert. Sie enthält Angaben über den internen Aufbau der Kamera, wie Bildsensorgröße und Brennweite, sowie die Position der Kamera in Relation zu der realen Umgebung. Dadurch lassen sich aus einem aufgenommenen Bild Informationen über die dreidimensionale Welt berechnen. 

Das Ziel der Kalibirierung einer Kamera ist es, eine Kameramatrix zu berechnen, die eine möglichst exakte Relation von Kamerabildern zu der realen Umgebung ermöglicht. Diese Kameramatrix kann mithilfe von Kalibriermustern erstellt werden. Dazu werden mehrere Fotos aufgenommen, auf denen das Muster in verschiedenen Entfernungen und Orientierungen zu sehen ist. Schließlich lässt sich aus diesen Daten die Kameramatrix berechnen.

Dieser Vorgang ist sehr zeitaufwändig. Zum einen müssen viele Bilder gemacht werden, um eine möglichst präzise Kalibrierung zu erhalten. Daher werden bis zu 50 Bilder aufgenommen. Zum anderen empfiehlt es sich, Kamera und Kalibiriermuster jeweils auf einem Stativ zu befestigen, um scharfe Bilder zu erzeugen. Damit ist gewährleistet, dass die Erkennung des Kalibriermusters nicht von unscharfen Bildern gestört wird, allerdings wird der Vorgang durch das Verstellen der Stative auch weiter verlängert.

Um die Kalibrierung zu vereinfachen, soll daher ein Roboterarm das Bewegen des Kalibriermusters übernehmen und der ganze Ablauf mit dem Erstellen der Bilder und dem Berechnen der Parameter durch ein Programm gesteuert werden. Der Benutzer platziert die Kamera auf einem Stativ vor dem Roboterarm, an dem das Kalibiriermuster befestigt ist. Anschließend startet er das Programm, welches den Arm an verschiedene Positionen bewegt, sodass die Kamera das Muster aus mehreren Entfernungen und Orientierungen aufnehmen kann. Zum Schluss berechnet das Programm die benötigten Parameter und der Benutzer kann die Kamera mit diesen Parametern einsetzen.